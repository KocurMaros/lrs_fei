V tomto projekte sme úspešne vytvorili autonómny systém na riadenie dronu s využitím ROS2. Podarilo sa nám implementovať kľúčové komponenty vrátane načítania máp a ich spracovania, algoritmov na vyhľadávanie ciest a optimalizáciu trajektórie, ako aj vytvorenie ROS uzlov pre komunikáciu a kontrolu nad dronom.

Načítanie a inflácia prekážok boli realizované efektívne, čím sme zabezpečili bezpečný pohyb dronu v zložitom prostredí. Plánovanie trajektórie pomocou algoritmu A* umožnilo efektívne a presné navigovanie medzi bodmi, pričom boli odstránené nadbytočné body, čím sa zlepšila efektivita letu. Funkcie \texttt{set\_arm} a \texttt{set\_mode} poskytli spoľahlivý spôsob na armovanie dronu a zmenu jeho režimu, čo je dôležité pre bezpečnosť a flexibilitu pri vykonávaní úloh.

Celkovo sme dosiahli funkčný a spoľahlivý autonómny navigačný systém, ktorý môže byť základom pre ďalšie vylepšenia a rozšírenia, ako je integrácia pokročilých senzorov alebo optimalizácia algoritmov pre ešte efektívnejšie vyhľadávanie ciest. Tento projekt predstavuje významný krok vpred v oblasti autonómnej robotiky a môže nájsť využitie v rôznych aplikáciách, kde je potrebná autonómna navigácia v reálnom čase.